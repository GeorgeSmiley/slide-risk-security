\documentclass[xcolor ={table,usenames,dvipsnames}]{beamer}
\usepackage[italian]{babel}
\usepackage{listings}
\PassOptionsToPackage{dvipsnames}{xcolor}
\title{Chapter 14: IT Security Management}

\author{Authors: Tommaso Puccetti, Edoardo Dini}
\institute{Universit\`a  degli Studi di Firenze}
\date{11/12/2018}
%\usepackage{sansmathaccent}
\usetheme{Berlin} 
\useinnertheme{rounded}
\useoutertheme{miniframes} 
\setbeamercovered{dynamic}
 
\theoremstyle{definition}
\newtheorem{definizione}{Definizione}





\begin{document}
	
	\begin{frame}
		\maketitle
	\end{frame}
	
		\begin{frame}
		\frametitle{Index}
		\begin{itemize}
			\item  \textbf{IT Security Management: overview}
			\begin{itemize}
				\item Overview
				\item Evolution and consensus
			\end{itemize}
			\item   \textbf{Organizational Context and Security Policy;}
			\begin{itemize}
				\item Security Objectives, Strategy and Policy
			\end{itemize}
			\item   \textbf{Approaches in Risk Assessment;}
			\begin{itemize}
				\item Baseline, Informal and Detailed Approaches
				\item Combined Approach
			\end{itemize}
			\item  \textbf{Detailed Security Risk Analysis};
			\begin{itemize}
				\item Establishing the Context
				\item Asset, Threat and Vulnerability identification
				\item Analyze Existing Security Control
				\item Risk Likelihood and Concequences
				\item Risk Level Determination and Meaning
				\item Risk Treatment
			\end{itemize}
			\item  \textbf{Case study: Silver Star Mine.}
		\end{itemize}
	\end{frame}
		
	\begin{frame}
		\frametitle{Index}
		\begin{itemize}
			\item <1-> \textbf{IT Security Management: overview}
			\begin{itemize}
				\item Overview
				\item Evolution and consensus
			\end{itemize}
			\item <0>  \textbf{Organizational Context and Security Policy;}
			\begin{itemize}
				\item Security Objectives, Strategy and Policy
			\end{itemize}
			\item <0>  \textbf{Approaches in Risk Assessment;}
			\begin{itemize}
				\item Baseline, Informal and Detailed Approaches
				\item Combined Approach
			\end{itemize}
			\item <0> \textbf{Detailed Security Risk Analysis};
			\begin{itemize}
				\item Establishing the Context
				\item Asset, Threat and Vulnerability identification
				\item Analyze Existing Security Control
				\item Risk Likelihood and Concequences
				\item Risk Level Determination and Meaning
				\item Risk Treatment
			\end{itemize}
			\item <0> \textbf{Case study: Silver Star Mine.}
		\end{itemize}
	\end{frame}

	
	\begin{frame}
		\frametitle{IT Security Management: overview (1)}
	\textbf{	Is the formal process of answering the questions:}
		\begin{figure}[h!]
			\centering
			\includegraphics[scale=0.20]{img/img_01.PNG}
			\label{Interfacce di un CS}
		\end{figure}
		\begin{itemize}
			\item Ensure that assets are sufficiently protected in a cost-effective manner;
			\item IT security risk assessment is needed for each asset in the organization that require protection;
			\item Provide the informations necessary to decide what management, operational and technical controls are needed to reduce the risks identified.
		\end{itemize}
	\end{frame}

	\begin{frame}
		\frametitle{IT Security Management: overview (2)}
		\begin{alertblock}{Definition:}
			 IT security management is a process used to achieve and maintain appropriate levels of confidentiality, integrity, availability, accountability, authenticity, and reliability. 
		\end{alertblock}
		\begin{itemize}
			\item Determining organizational IT security objectives, strategies, and policies;
			\item Determining organizational IT security requirements;
			\item Identifying and analyzing security threats to IT assets; 
			\item Identifying and analyzing risks;
			\item Specifying appropriate safeguards;
			\item Monitoring the implementation and operation of safeguards;
			\item Developing and implementing a security awareness program;
			\item Detecting and reacting to incidents.
		\end{itemize}	
	\end{frame}

	\begin{frame}
		\frametitle{IT Security Management: a cyclic process (1)}
	    \textbf{It is important to emphasize that:}
	    \begin{itemize}
	    	\item IT security management needs to be a key part of an organization’s overall management plan;
	    	\item IT security risk assessment process should be incorporated into the
	    	wider risk assessment of all the organization’s assets and business processes;
	    	\item IT security management is a cyclic process \textbf{that must be repeated constantly} (as specified in [ISO27001]).
	    \end{itemize} 
	\end{frame}
	
	\begin{frame}
		\frametitle{IT Security Management: a cyclic process (2)}
			\begin{figure}[h!]
			\centering
			\includegraphics[scale=0.50]{img/img_02.PNG}
			\label{Interfacce di un CS}
		\end{figure}
	\end{frame}
	
	\begin{frame}
		\frametitle{IT Security Management: a cyclic process (3)}
			\begin{figure}[h!]
			\centering
			\includegraphics[scale=0.40]{img/img_03.PNG}
			\label{Interfacce di un CS}
		\end{figure}
	\end{frame}
	
	\begin{frame}
		\frametitle{Evolution and consensus}
		The discipline of IT security management has evolved considerably over the last few decades:
		 
		\begin{itemize}
			\item This has occurred in response to the rapid growth of, and dependence on, networked computer systems and the associated rise in risks to these systems;
			\item In the last	decade a number of national and international standards have been published. These represent a consensus on the best practice in the field;
			\item \textbf{The International Standards Organization (ISO) has revised and consolidated a number of these standards into the 27000 series.}
		\end{itemize}
		 
	\end{frame}

	\begin{frame}
		\frametitle{ISO 27000 series of Standards on IT Security Techniques }
		\begin{table}[]
			\resizebox{11.5cm}{!}{
				\begin{tabular}{|
					>{\columncolor[HTML]{32CB00}}l |l|}
				\hline
				27000:2012 & \begin{tabular}[c]{@{}l@{}}\textquotedblleft Information security management systems: Overview and vocabulary\textquotedblright provides an\\ ­overview of information security management systems, and defines the vocabulary and\\ ­definitions used in the 27000 family of standards.\end{tabular}              \\ \hline
				27001:2005 & \begin{tabular}[c]{@{}l@{}}\textquotedblleft Information security management systems: Requirements\textquotedblright specifies the requirements for\\ establishing, implementing, operating, monitoring, reviewing, maintaining, and improving a\\ documented Information Security Management System.\end{tabular} \\ \hline
				27002:2005 & \begin{tabular}[c]{@{}l@{}}\textquotedblleft Code of practice for information security management\textquotedblright provides guidelines for informa-\\ tion security management in an organization and contains a list of best-practice security\\ controls. It was formerly known as ISO17799.\end{tabular}      \\ \hline
				27003:2010 & \begin{tabular}[c]{@{}l@{}}\textquotedblleft Information security management system implementation guidance\textquotedblright details the process\\ from inception to the production of implementation plans of an Information Security\\ Management System specification and design.\end{tabular}                \\ \hline
				27004:2009 & \begin{tabular}[c]{@{}l@{}}\textquotedblleft Information security management: Measurement\textquotedblright provides guidance to help organiza-\\ tions measure and report on the effectiveness of their Information Security Management\\ System processes and controls.\end{tabular}                             \\ \hline
				27005:2011 & \begin{tabular}[c]{@{}l@{}}\textquotedblleft Information security risk management\textquotedblright provides guidelines on the information security\\ risk management process. It supersedes ISO13335-3/4.\end{tabular}                                                                                           \\ \hline
				27006:2007 & \begin{tabular}[c]{@{}l@{}}\textquotedblleft Requirements for bodies providing audit and certification of information security\\ ­management systems\textquotedblright specifies requirements and provides guidance for these bodies.\end{tabular}                                                                \\ \hline
			\end{tabular}
		}
		\end{table}
	\end{frame}
	\begin{frame}
		\frametitle{Index}
		\begin{itemize}
			\item <0> \textbf{IT Security Management: overview}
			\begin{itemize}
				\item Overview
				\item Evolution and consensus
			\end{itemize}
			\item <1->  \textbf{Organizational Context and Security Policy;}
			\begin{itemize}
				\item Security Objectives, Strategy and Policy
			\end{itemize}
			\item <0>  \textbf{Approaches in Risk Assessment;}
			\begin{itemize}
				\item Baseline, Informal and Detailed Approaches
				\item Combined Approach
			\end{itemize}
			\item <0> \textbf{Detailed Security Risk Analysis};
			\begin{itemize}
				\item Establishing the Context
				\item Asset, Threat and Vulnerability identification
				\item Analyze Existing Security Control
				\item Risk Likelihood and Concequences
				\item Risk Level Determination and Meaning
				\item Risk Treatment
			\end{itemize}
			\item <0> \textbf{Case study: Silver Star Mine.}
		\end{itemize}
	\end{frame}

	\begin{frame}
		\frametitle{Organizational Context and Security Policy}
		\begin{figure}[h!]
			\centering
			\includegraphics[scale=0.20]{img/img_04.PNG}
			\label{Interfacce di un CS}
		\end{figure}
		In IT security management process comprises an examination of the organization's IT security \textbf{object, strategies and policies}. This can only occur in the wider context of the organization's management.
	
	\end{frame}

	\begin{frame}
		\frametitle{Security Objectives}
		\begin{itemize}
			\item What key aspects of the organization require IT support in order to function efficiently?
			\item What tasks can only be performed with IT support?
			\item Which essential decisions depend on the accuracy, currency, integrity, or availability of data managed by the IT systems?
			\item What data created, managed, processed, and stored by the IT systems need
			protection?
			\item What are the consequences to the organization of a security failure in their IT systems?
		\end{itemize}
	\end{frame}

	\begin{frame}
		\frametitle{Security Strategy}
		Once the objectives are listed, some broad strategy statements can be developed.
		These outline, in general terms, \textbf{how the identified objectives will be met} in a consistent manner across the organization:
		\begin{itemize}
			 \item The topics and details in the strategy ­statements
			depend on the identified objectives, the size of the organization, and the importance of the IT systems to the organization; 
			\item The strategy statements should address the approaches the organization will use to manage the security of its IT systems. 
		\end{itemize}	
	\end{frame}

	\begin{frame}
		\frametitle{Security Policy (1)}
		Given the organizational security objectives and strategies, an organizational security policy is developed that describes what the objectives and strategies are and the process used to achieve them. 
		\begin{itemize}
			\item The scope and purpose of the policy;
			\item The relationship of the security objectives to the organization’s legal and regulatory obligations, and its business objectives;
			\item IT security requirements in terms of confidentiality, integrity, availability, accountability, authenticity, and reliability, particularly with regard to the views of the asset owners;
			\item The assignment of responsibilities relating to the management of IT security and the organizational infrastructure;
			\item The risk management approach adopted by the organization;
		\end{itemize}
	\end{frame}

	\begin{frame}
		\frametitle{Security Policy (2)}
		
		\begin{itemize}
		\item How security awareness and training is to be handled;
		\item General personnel issues, especially for those in positions of trust;
		\item Any legal sanctions that may be imposed on staff, and the conditions under which such penalties apply;
		\item Integration of security into systems development and procurement;
		\item Definition of the information classification scheme used across the organization;
		\item Contingency and business continuity planning;
		\item Incident detection and handling processes;
		\item How and when this policy should be reviewed;
		\item The method for controlling changes to this policy.
		\end{itemize}
	\end{frame}

	\begin{frame}
		\frametitle{Management Support}
		IT security policy must be supported by senior management. \\
		Need IT security officer:
		\begin{itemize}
			\item To provide consistent overall supervision;
			\item Connection with senior management;
			\item Maintenance of IT security objectives, strategies, policies;
			\item Handle incidents;
			\item Management of IT security awareness and training programs;
			\item Interaction with IT project security officers.
		\end{itemize}
		Large organizations need separate IT project security officers associated with major projects and systems.
	\end{frame}
	
	\begin{frame}
		\frametitle{Index}
		\begin{itemize}
			\item <0> \textbf{IT Security Management: overview}
			\begin{itemize}
				\item Overview
				\item Evolution and consensus
			\end{itemize}
			\item <0>  \textbf{Organizational Context and Security Policy;}
			\begin{itemize}
				\item Security Objectives, Strategy and Policy
			\end{itemize}
			\item <1->  \textbf{Approaches in Risk Assessment;}
			\begin{itemize}
				\item Baseline, Informal and Detailed Approaches
				\item Combined Approach
			\end{itemize}
			\item <0> \textbf{Detailed Security Risk Analysis};
			\begin{itemize}
				\item Establishing the Context
				\item Asset, Threat and Vulnerability identification
				\item Analyze Existing Security Control
				\item Risk Likelihood and Concequences
				\item Risk Level Determination and Meaning
				\item Risk Treatment
			\end{itemize}
			\item <0> \textbf{Case study: Silver Star Mine.}
		\end{itemize}
	\end{frame}
	
	\begin{frame}
		\frametitle{Security Risk Assessment}
		\begin{figure}[h!]
			\centering
			\includegraphics[scale=0.25]{img/img_05.PNG}
			\label{Interfacce di un CS}
		\end{figure}
	\end{frame}

		
	\begin{frame}
		\frametitle{Baseline Approach}
			\begin{figure}[h!]
				\centering
				\includegraphics[scale=0.23]{img/img_06.PNG}
				\label{Interfacce di un CS}
			\end{figure}
			%\begin{itemize} 
			%	\item Goal is to implement agreed controls to provide protection against the most common threats
			%	\item Forms a good base for further security measures
			%	\item Use “industry best practice”
			%	\begin{itemize}
			%		\item Easy, cheap, can be replicated
			%		\item Gives no special consideration to variations in risk exposure
			%		\item May give too much or too little security
			%	\end{itemize}
			%	\item Generally recommended only for small organizations without the resources to implement more structured approaches
			%\end{itemize}
	\end{frame}
	
	\begin{frame}
		\frametitle{Informal Approach}
		\begin{figure}[h!]
			\centering
			\includegraphics[scale=0.27]{img/img_07.PNG}
			\label{Interfacce di un CS}
		\end{figure}
	\end{frame}

	\begin{frame}
		\frametitle{Detailed Risk Analysis}
		\begin{figure}[h!]
			\centering
			\includegraphics[scale=0.23]{img/img_08.PNG}
			\label{Interfacce di un CS}
		\end{figure}
	\end{frame}

%	\begin{frame}
%		\frametitle{Combined Approach (1)} 
%		 Aim is to provide reasonable levels of protection as quickly as possible then to examine and adjust the protection controls deployed on key systems over time:
%		Over time, this can result in the most appropriate and cost-effective security %controls being selected and implemented on these systems
%	\end{frame}
	
	\begin{frame}
		\frametitle{Combined Approach}
		\begin{enumerate}
			\item This approach starts with the implementation of suitable baseline security recommendations on all systems;
			\item Next, systems either exposed to high risk levels or critical to the organization's business objectives are identified in the high-level risk assessment;
			\item A decision can then be made to possibly conduct an immediate informal risk assessment on key systems, with the aim of relatively quickly tailoring controls to more accurately reflect their requirements;
			\item Lastly, an ordered process of performing detailed risk analyses of these systems can be instituted.
		\end{enumerate}
	\end{frame}
	
	\begin{frame}
		\frametitle{Index}
		\begin{itemize}
			\item <0> \textbf{IT Security Management: overview}
			\begin{itemize}
				\item Overview
				\item Evolution and consensus
			\end{itemize}
			\item <0>  \textbf{Organizational Context and Security Policy;}
			\begin{itemize}
				\item Security Objectives, Strategy and Policy
			\end{itemize}
			\item <0>  \textbf{Approaches in Risk Assessment;}
			\begin{itemize}
				\item Baseline, Informal and Detailed Approaches
				\item Combined Approach
			\end{itemize}
			\item <1-> \textbf{Detailed Security Risk Analysis};
			\begin{itemize}
				\item Establishing the Context
				\item Asset, Threat and Vulnerability identification
				\item Analyze Existing Security Control
				\item Risk Likelihood and Concequences
				\item Risk Level Determination and Meaning
				\item Risk Treatment
			\end{itemize}
			\item <0> \textbf{Case study: Silver Star Mine.}
		\end{itemize}
	\end{frame}
	
	\begin{frame}
		\frametitle{Detailed Security Risk Analysis (1)}
		\begin{figure}[h!]
			\centering
			\includegraphics[scale=0.27]{img/img_09.PNG}
			\label{Interfacce di un CS}
		\end{figure}
		\end{frame}
	
	\begin{frame}
		\frametitle{Detailed Security Risk Analysis (2)}
		\begin{figure}[h!]
			\centering
			\includegraphics[scale=0.47]{img/img_10.PNG}
			\label{Interfacce di un CS}
		\end{figure}
	\end{frame}
	
	\begin{frame}
		\frametitle{Establishing the Context (1)}
		\begin{itemize}
			\item	Initial steps: 
			\begin{itemize}
				\item Determine the basic parameters of the risk assessment;
				\item Identify the assets to be examined.
			\end{itemize}
			\item Explores political and social environment in which the organization operates: \begin{itemize}
				\item 	Legal and regulatory constraints;
				\item	Provide baseline for organization’s risk exposure.
			\end{itemize}
			\item The \textbf{risk appetite} is the level of risk the organization views as acceptable.
		\end{itemize}	
	\end{frame}

	\begin{frame}
		\frametitle{Establishing the Context (2)}
		\begin{figure}[h!]
			\centering
			\includegraphics[scale=0.45]{img/img_11.PNG}
			\label{Interfacce di un CS}
		\end{figure}
	\end{frame}

	\begin{frame}
		\frametitle{Terminology}
			\begin{itemize}
				\item \textbf{Asset}: a system resource or capability of value to its owner that requires protection; 
				\item \textbf{Threat}: a potential for a threat source to exploit a vulnerability in some asset, which if it occurs may compromise the security of the asset and cause harm to the asset’s owner;
				\item \textbf{Vulnerability}: a flaw or weakness in an asset’s design, implementation, or operation and management that could be exploited by some threat;
				\item \textbf{Risk}: The potential for loss, computed as the 	combination of the likelihood that a given threat exploits some vulnerability to an asset, and the magnitude of harmful consequence that results to the asset’s owner.
			\end{itemize}
	\end{frame}

	\begin{frame}
		\frametitle{Asset Identification}
		\textit{\textbf{"What assets we need to protect?"}}
		\begin{itemize}
			\item These assets need to be identified and their value to the organization assessed;
			\item The ideal is to consider every conceivable asset, in practice this is not possible. Rather the goal here is to identify all assets that contribute significantly to attaining the organization’s objectives and whose compromise the organization’s operation; 
			\item A security expert may not have an high knowledge of the organization's operation and structure, so experts for each area of the organization are needed for the process.
		\end{itemize}
	\end{frame}
	
	\begin{frame}
		\frametitle{Threat Identification (1)}
			\begin{figure}[h!]
			\centering
			\includegraphics[scale=0.25]{img/img_12.PNG}
			\label{Interfacce di un CS}
		\end{figure}
	\end{frame}

	\begin{frame}
		\frametitle{Threat Identification (2)}
		\textbf{\textit{"Who or what could cause harm to the assets?"}}
		\begin{itemize}
			\item Identifying possible threats and threat sources requires the use of a \textbf{variety of sources}, along with the experience of the risk assessor; 
			\item  Organization’s define threat scenarios to describe how the tactics, techniques, and procedures employed by an attacker can contribute to, or cause, harm. 
		\end{itemize}
	\end{frame}

	\begin{frame}
		\frametitle{Threat Identification (3)}
		\begin{enumerate}
			\item  \textbf{Motivation}: Why would they target this organization; how motivated are they?
			\item \textbf{Capability}: What is their level of skill in exploiting the threat?
			\item \textbf{Resources}: How much time, money, and other resources could they deploy?
			\item \textbf{Probability of attack}: How likely and how often would your assets be targeted?
			\item \textbf{Deterrence}: What are the consequences to the attacker of being identified?	
		\end{enumerate}
	\end{frame}
	
	\begin{frame}
		\frametitle{Vulnerability Identification}
		 \textbf{"How could this occur?"}
		\begin{itemize}
			\item \textbf{Identify} exploitable flaws or weaknesses in organization’s IT systems or processes:
			\begin{itemize}
				\item Determines applicability and significance of threat to organization.
			\end{itemize}
			\item Need combination of threat and vulnerability to create a risk to an asset;
			\item \textbf{Outcome} should be a list of threats and vulnerabilities with brief descriptions of how and why they might occur.
		\end{itemize}
	\end{frame}
	
	\begin{frame}
		\frametitle{Analyze Risks}
		\begin{itemize}
			\item Specify likelihood of occurrence of each identified threat to asset, given existing controls;
			\item Specify consequence should threat occur;
			\item Hard to determine accurate probabilities and realistic cost consequences;
			\item Use qualitative, not quantitative, ratings; 
			\item Derive overall risk rating for each threat.
		\end{itemize}
		\begin{alertblock}{Definition:}
			\textbf{Risk = Probability threat occurs x Cost to organization}
		\end{alertblock}
	\end{frame}

	\begin{frame}
		\frametitle{Analyze Existing Security Control } 
		\begin{itemize}
			\item Existing controls used to attempt to minimize threats need to be identified;
			\item Security controls include:
			\begin{enumerate}
				\item Management;
				\item Operational;
				\item Technical processes and procedures.
			\end{enumerate}
			\item Use checklists of existing controls and interview key organizational staff to solicit information.
		\end{itemize}
	\end{frame}

	\begin{frame}
		\frametitle{Risk Likelihood (1)}
		\begin{itemize}
			\item Take the assets and the threath/vulnerability from the previous steps and decides an appropriate rating. It is related to the \textbf{likelihood of a specified threath exploiting one or more vulnerability to an asset};
			\item When deliberate \textbf{human-made threat} sources are considered, this
			estimate should include an evaluation of the attackers intent, capability, and specific targeting of this organization (an high rating suggests that a threat has occurred sometimes previously);
		\end{itemize}
	\end{frame}

	\begin{frame}
		\frametitle{Risk Likelihood (2)}
		\begin{itemize}
			\item There will very likely be some uncertainty and debate over exactly which rating is most appropriate. The final decision will be taken by the management;
			\item The likelihood is typically described \textbf{qualitatively}, using values and descriptions such as those shown in the table.
		\end{itemize}
		\begin{table}[]
			\resizebox{10cm}{!}{
				\begin{tabular}{|l|l|l|}
					\hline
					\rowcolor[HTML]{32CB00} 
					Rating & \begin{tabular}[c]{@{}l@{}}Likelihood\\ Description\end{tabular} & Expanded Definition                                                                                                                                                       \\ \hline
					1      & Rare                                                             & \begin{tabular}[c]{@{}l@{}}May occur only in exceptional circumstances and may be deemed a\\ “unlucky” or very unlikely.\end{tabular}                                     \\ \hline
					2      & Unlikely                                                         & \begin{tabular}[c]{@{}l@{}}Could occur at some time but not expected given current \\ controls, ­circumstances, and recent events.\end{tabular}                           \\ \hline
					3      & Possible                                                         & \begin{tabular}[c]{@{}l@{}}Might occur at some time, but just as likely as not. It may be difficult \\ to control its occurrence due to external influences.\end{tabular} \\ \hline
					4      & Likely                                                           & \begin{tabular}[c]{@{}l@{}}Will probably occur in some circumstance and one should \\ not be ­surprised if it occurred.\end{tabular}                                      \\ \hline
					5      & Almost Certain                                                   & Is expected to occur in most circumstances and certainly sooner.                                                                                                           \\ \hline
				\end{tabular}
			}
		\end{table}
	\end{frame}

	\begin{frame}
		\frametitle{Risk Consequences (1)}
		\begin{itemize}
			\item Consequence specification indicates the impact on the organization (and not only on the security system) should the particular threat in question actually eventuate;
			\item Use \textbf{qualitative values};
			\item As with the likelihood ratings, there is likely to be some \textbf{uncertainty} as to the best rating to use;
			\item As with the likelihood ratings, the consequence ratings must be determined \textbf{knowing the organization’s current practices} (existing backup, disaster recovery, and contingency planning), will influence the choice of rating.
		\end{itemize}
	\end{frame}
	
	
	\begin{frame}
		\frametitle{Risk Consequences (2)}
		\begin{table}[]
				\resizebox{10cm}{!}{
			\begin{tabular}{|l|l|l|}
				\hline
				\rowcolor[HTML]{32CB00} 
				Rating & Consequence   & Expanded Definition                                                                                                                                                                                                                                                                                                                                                                                                                                     \\ \hline
				1      & Insignificant & \begin{tabular}[c]{@{}l@{}}Generally a result of a minor security breach in a single area. Impact\\ is likely to last less than several days and requires only minor expendi-\\ ture to rectify. Usually does not result in any tangible detriment to the\\ organization.\end{tabular}                                                                                                                                                                  \\ \hline
				2      & Minor         & \begin{tabular}[c]{@{}l@{}}Result of a security breach in one or two areas. Impact is likely to last\\ less than a week but can be dealt with at the segment or project level\\ without management intervention. Can generally be rectified within\\ project or team resources. Again, does not result in any tangible det-\\ riment to the organization, but may, in hindsight, show previous lost\\ opportunities or lack of efficiency.\end{tabular} \\ \hline
				3      & Moderate      & \begin{tabular}[c]{@{}l@{}}Limited systemic (and possibly ongoing) security breaches. Impact\\ is likely to last up to 2 weeks and will generally require manage-\\ ment intervention, though should still be able to be dealt with at the\\ project or team level. Will require some ongoing compliance costs to\\ overcome. Customers or the public may be indirectly aware or have\\ limited information about this event.\end{tabular}              \\ \hline
			\end{tabular}
		}
		\end{table}
	\end{frame}

	\begin{frame}
		\frametitle{Risk Consequences (3)}
		\begin{table}[]
			\resizebox{8.5cm}{!}{
			\begin{tabular}{|l|l|l|}
				\hline
				\rowcolor[HTML]{32CB00} 
				Rating & Consequence  & Expanded Definition                                                                                                                                                                                                                                                                                                                                                                                                                                                                                                                                                                                                 \\ \hline
				4      & Major        & \begin{tabular}[c]{@{}l@{}}Ongoing systemic security breach. Impact will likely last 4–8 weeks\\ and require significant management intervention and resources to\\ overcome. Senior management will be required to sustain ongoing\\ direct management for the duration of the incident and compliance\\ costs are expected to be substantial. Customers or the public will be\\ aware of the occurrence of such an event and will be in possession\\ of a range of important facts. Loss of business or organizational out-\\ comes is possible, but not expected, especially if this is a once off.\end{tabular} \\ \hline
				5      & Catastrophic & \begin{tabular}[c]{@{}l@{}}Major systemic security breach. Impact will last for 3 months or\\ more and senior management will be required to intervene for the\\ duration of the event to overcome shortcomings. Compliance costs\\ are expected to be very substantial. A loss of customer business or\\ other significant harm to the organization is expected. Substantial\\ public or political debate about, and loss of confidence in, the orga-\\ nization is likely. Possible criminal or disciplinary action against\\ personnel involved is likely.\end{tabular}                                          \\ \hline
				6      & Doomsday     & \begin{tabular}[c]{@{}l@{}}Multiple instances of major systemic security breaches. Impact dura-\\ tion cannot be determined and senior management will be required\\ to place the company under voluntary administration or other form\\ of major restructuring. Criminal proceedings against senior man-\\ agement is expected, and substantial loss of business and failure to\\ meet organizational objectives is unavoidable. Compliance costs are\\ likely to result in annual losses for some years, with liquidation of\\ the organization likely.\end{tabular}                                              \\ \hline
			\end{tabular}
		}
		\end{table}
	\end{frame}
	
	\begin{frame}
		\frametitle{Risk level determination and meaning}
		\begin{table}[]
			\resizebox{10cm}{!}{
			\begin{tabular}{|l|l|l|l|l|l|l|}
				\hline
				\rowcolor[HTML]{32CB00} 
				& \multicolumn{6}{l|}{\cellcolor[HTML]{32CB00}\textbf{Consequences}} \\ \hline
				\rowcolor[HTML]{32CB00} 
				\cellcolor[HTML]{32CB00}\textbf{Likelihood} & Doomsday & Catastrophic & Major & Moderate & Minor & Insignificant \\ \hline
				\cellcolor[HTML]{32CB00}Almost Certain      & E        & E            & E     & E        & H     & H             \\ \hline
				\cellcolor[HTML]{32CB00}Likely              & E        & E            & E     & H        & H     & M             \\ \hline
				\cellcolor[HTML]{32CB00}Possible            & E        & E            & E     & H        & M     & L             \\ \hline
				\cellcolor[HTML]{32CB00}Unlikely            & E        & E            & H     & M        & L     & L             \\ \hline
				\cellcolor[HTML]{32CB00}Rare                & E        & H            & H     & M        & L     & L             \\ \hline
			\end{tabular}
		}
		\end{table}
	\begin{table}[]
		\resizebox{10cm}{!}{
		\begin{tabular}{|l|l|}
			\hline
			\rowcolor[HTML]{32CB00} 
			Risk Level  & Description                                                                                                                                                                                                                                                                                                                        \\ \hline
			Extreme (E) & \begin{tabular}[c]{@{}l@{}}Will require detailed research and management planning at an executive/director level.\\ Ongoing planning and monitoring will be required with regular reviews. \\ Substantial adjustment of controls to manage the risk is expected, with costs possibly \\ exceeding original forecasts.\end{tabular} \\ \hline
			High (H)    & \begin{tabular}[c]{@{}l@{}}Requires management attention, but management and planning can be left to senior\\ project or team leaders. Ongoing planning and monitoring with regular reviews are\\ likely, though adjustment of controls is likely to be met from within existing resources.\end{tabular}                           \\ \hline
			Medium (M)  & \begin{tabular}[c]{@{}l@{}}Can be managed by existing specific monitoring and response procedures. \\ Management by employees is suitable with appropriate monitoring and reviews.\end{tabular}                                                                                                                                    \\ \hline
			Low (L)     & Can be managed through routine procedures.                                                                                                                                                                                                                                                                                          \\ \hline
		\end{tabular}
	}
	\end{table}
	\end{frame}

	\begin{frame}
		\frametitle{Risk Register}
		\begin{itemize}
			\item Used to \textbf{keep track of the risk analysis process} and, if needed, provides evidence that a formal risk assessment process has been followed;
			\item The risks are usually sorted in decreasing
			order of level. This would be supported by details of how the various items were determined, including the rationale, justification, and and supporting evidence used.
		\end{itemize}
		\begin{table}[]
				\resizebox{10cm}{!}{
			\begin{tabular}{|l|l|l|l|l|l|l|}
				\hline
				\rowcolor[HTML]{32CB00} 
				Asset                                                                  & \begin{tabular}[c]{@{}l@{}}Threat /\\ Vulnerability\end{tabular}    & \begin{tabular}[c]{@{}l@{}}Existing \\ Controls\end{tabular}                  & Likelihood & Consequence & Level of Risk & Risk Priority \\ \hline
				\begin{tabular}[c]{@{}l@{}}Internet\\ router\end{tabular}              & \begin{tabular}[c]{@{}l@{}}Outside Hacker \\ attack\end{tabular}    & \begin{tabular}[c]{@{}l@{}}Admin \\ password only\end{tabular}                & Possible   & Moderate    & High          & 1             \\ \hline
				\begin{tabular}[c]{@{}l@{}}Destruction\\ of data\\ center\end{tabular} & \begin{tabular}[c]{@{}l@{}}Accidental fire \\ or flood\end{tabular} & \begin{tabular}[c]{@{}l@{}}None (no\\ disaster \\ recovery plan)\end{tabular} & Unlikely   & Major       & High          & 2             \\ \hline
			\end{tabular}
		}
		\end{table}
	\end{frame}

	\begin{frame}
		\frametitle{Risk Treatment (1)}	
		\begin{itemize}
			\item Management decides which treatments should be apllied, considering the data collected in the previous phases;
			\item The decision is conditioned by two factors:
			\begin{enumerate}
				\item Level of risk;
				\item Cost for treatment implementation.
			\end{enumerate}
			\item If the cost of treatment is high, but the risk is low, then it is usually uneconomic to proceed with such treatment; 
			\item If the risk is high and the cost is comparatively low, treatment should occur;
			\item The most difficult area occurs between these extremes. 
		\end{itemize}
	\end{frame}

	\begin{frame}
		\frametitle{Risk Treatment (2)}
		\begin{figure}[h!]
			\centering
			\includegraphics[scale=0.55]{img/img_13.PNG}
			\label{Interfacce di un CS}
		\end{figure}
	\end{frame}
	
	\begin{frame}
		\frametitle{Risk Treatment (3)}
		\begin{figure}[h!]
			\centering
			\includegraphics[scale=0.25]{img/img_14.PNG}
			\label{Interfacce di un CS}
		\end{figure}
	\end{frame}
	
	\begin{frame}
		\frametitle{Index}
		\begin{itemize}
			\item <0> \textbf{IT Security Management: overview}
			\begin{itemize}
				\item Overview
				\item Evolution and consensus
			\end{itemize}
			\item <0>  \textbf{Organizational Context and Security Policy;}
			\begin{itemize}
				\item Security Objectives, Strategy and Policy
			\end{itemize}
			\item <0>  \textbf{Approaches in Risk Assessment;}
			\begin{itemize}
				\item Baseline, Informal and Detailed Approaches
				\item Combined Approach
			\end{itemize}
			\item <0> \textbf{Detailed Security Risk Analysis};
			\begin{itemize}
				\item Establishing the Context
				\item Asset, Threat and Vulnerability identification
				\item Analyze Existing Security Control
				\item Risk Likelihood and Concequences
				\item Risk Level Determination and Meaning
				\item Risk Treatment
			\end{itemize}
			\item <1> \textbf{Case study: Silver Star Mine.}
		\end{itemize}
	\end{frame}
	
	\begin{frame}
		\frametitle{Case Study: Silver Star Mine (1)}
		A case study involving the operations of  company Silver Star Mines illustrates this risk assessment process. 
		\begin{itemize}
			\item Silver Star Mines is the local operations of a large global
			mining company;
			\item It has a large IT infrastructure;
			\item Its network includes a variety of servers, executing a range of ­application ­software;
			\item It uses also applications directly related to the health and safety of those working in the mine;
			\item Many of these systems used to be isolated, with no network connections among them;
			\item In recent years, they have been connected together and connected to the ­company’s intranet to provide better management capabilities. 
		\end{itemize}
	\end{frame}

	\begin{frame}
		\frametitle{Case Study: Silver Star Mine (2)}
		The security analyst and company management decided to adopt a \textbf{combined approach}:
		\begin{itemize}
			\item The analyst was asked to conduct a \textbf{preliminary} formal assessment of the key IT systems;
			\item The \textbf{context} for the risk assessment was determined: being in the mining industry sector places the company at the less risky end of the ­spectrum;
			\item SSM is part of a large organization and hence is subject to legal requirements for occupational health and safety. Thus management \textbf{decided to accept only moderate or lower risks.}
			%SEMMAI AGGIUNGERE that it wished to accept only moderate or lower risks in general. The boundaries
			%for this risk assessment were specified to include only the systems under the direct
			%control of the Silver Star Mines operations. This excluded the wider company
			%intranet, its central servers, and its Internet gateway. This assessment is sponsored
			%by Silver Star’s IT and engineering managers, with results to be reported to the
		\end{itemize}
	\end{frame}

	\begin{frame}
		\frametitle{Assets Identification (1)}
		The analyst conducted interviews with key IT and engineering managers in the company, identifying the following key assets:
		\begin{itemize}
			\item \textbf{SCADA Network};
			\item \textbf{Data integrity};
			\item \textbf{Financial, Procurement and Maintenance/Production servers;}
			\item \textbf{Email service}.
		\end{itemize}
		Having determined the list of key assets, the analyst needed to identify significant threats to these assets and to specify the likelihood and consequence values.
	\end{frame}

	\begin{frame}
		\frametitle{Assets Identification (2)}
		\textbf{SCADA Network}:
		\begin{itemize}
			\item Control and monitor the core mining operations.
			\item Maintain the records required by law.
		\end{itemize}
		\textbf{Data integrity}:
		\begin{itemize}
			\item Data collected from different sources;
			\item Some of this data are required by law;
			\item Data on production and operational results are extremely valuable for the company.
		\end{itemize}
		\textbf{Financial, Procurement and Maintenance/Production servers}:
		\begin{itemize}
			\item Critical to the effective operation of core business area.
		\end{itemize}
		\textbf{Email service}:
		\begin{itemize}
			\item Used across all business areas.
			\item Greater importance given, due to the remote location of the company. 
		\end{itemize}
	\end{frame}

	\begin{frame}
		\frametitle{Level of Risk: SCADA Network}
		\begin{itemize}
			\item \textbf{Threat}: unauthorized compromise of nodes by external source.
			\item \textbf{Existing controls}: recently additional firewall and proxy service was introduce to connect the system to the intranet.
			\item \textbf{Likelihood: Rare}
			\begin{itemize}
				\item An external attack require a series of security breaches;
				\item Various computer crime surveys suggest that externally sourced attacks are increasing;
				\item The analyst concluded that while an attack was
				very unlikely, it could still occur.
			\end{itemize}
			\item \textbf{Consequences: Major}
			\begin{itemize}
				\item Serious consequences to the safety of mine's personnel;
				\item Significant financial impact (in downtime, 10 milions per hour).
				\item Breach of legal requirement.
			\end{itemize}
		\end{itemize}
	\end{frame}
	
	\begin{frame}
		\frametitle{Level of Risk: Data Integrity}
		\begin{itemize}
			\item \textbf{Threat}: compromise of data from internal or external sources, even malicious.
			\item \textbf{Existing controls}: 
			\begin{itemize}
				\item Company's intranet is shielded by the outer firewall;
				\item Policies on the input handling of data;
				\item Policies on data backup from server.
			\end{itemize} 
			\item \textbf{Likelihood: Possible }
			\begin{itemize}
				\item Overall compliance with the policies listed above is unknown.
				\item Computer crime surveys indicate these kind of data as primary goal of intruder. 
			\end{itemize}	
			\item \textbf{Consequences: Major}
			\begin{itemize}
				\item Financial harm due to the confidential nature of these data.
				\item Financial cost involved with the recover of data operations.
				\item Legal consequences in case of personal informations disclosure.
			\end{itemize}
		\end{itemize}
	\end{frame}

	\begin{frame}
		\frametitle{Level of Risk: Financial and Procurement Servers}
		\begin{itemize}
			\item \textbf{Threat}: any form of attack on perating systems or application they use.
			\item \textbf{Existing controls}: 
			\begin{itemize}
				\item Servers are placed in the company's intranet, thus are shielded by the outer firewall.
			\end{itemize} 
			\item \textbf{Likelihood: Possible }
			\begin{itemize}
				\item Any failure in company's outer firewall could very likely result in compromise of some systems;
				\item Security reports indicate that unpatched systems could be compromised in less than 15 minutes after network connection. 
			\end{itemize}	
			\item \textbf{Consequences: Moderate }
			\begin{itemize}
				\item Propotional to extent and duration of the attack;
				\item Rebuild of at least a portion of the system.
				\item False orders or inability to issue order.
				\item Inability of use electronic found. 
			\end{itemize}
		\end{itemize}
	\end{frame}

	\begin{frame}
		\frametitle{Level of Risk: Maintenance/Production Servers}
		\begin{itemize}
			\item \textbf{Threat}: any form of attack on perating systems or application they use.
			\item \textbf{Existing controls}: 
			\begin{itemize}
				\item Servers are placed in the company's intranet, thus are shielded by the outer firewall. 
			\end{itemize} 
			\item \textbf{Likelihood: Possible  }
			\begin{itemize}
				\item Any failure in company's outer firewall could very likely result in compromise of some systems;
				\item Security reports indicate that unpatched systems could be compromised in less than 15 minutes after network connection.  
			\end{itemize}	
			\item \textbf{Consequences: Major }
			\begin{itemize}
				\item Detrimental impact on the efficiency of operations;
				\item The systems are capable to operate despite some compromise of the systems.
			\end{itemize}
		\end{itemize}
	\end{frame}

	\begin{frame}
		\frametitle{Level of Risk: E-Mail Service}
		\begin{itemize}
			\item \textbf{Threat}: e-mailed worms and DoS attacks.
			\item \textbf{Existing controls}: 
			\begin{itemize}
				\item The company does filter e-mail in its Internet gateway. 
			\end{itemize} 
			\item \textbf{Likelihood: Almost Certain}
			\begin{itemize}
				\item The use of e-mail attachments could be used to compromise these systems;
				\item DoS attacks against the mail gateway is very hard to defend.
			\end{itemize}	
			\item \textbf{Consequences: Major }
			\begin{itemize}
				\item Financial costs and time to rebuild the e-mail system;
				\item Inability to send or receive reports may affect the company reputation.
				\item However the compromise of such system would not have a large impact on the company's operations.
			\end{itemize}
		\end{itemize}
	\end{frame}
	
	\begin{frame}
		\frametitle{Case Study: Risk Register}
		% Please add the following required packages to your document preamble:
		% \usepackage[table,xcdraw]{xcolor}
		% If you use beamer only pass "xcolor=table" option, i.e. \documentclass[xcolor=table]{beamer}
		\begin{table}[]
			\resizebox{8,3cm}{!}{
			\begin{tabular}{|l|l|l|l|l|l|l|}
				\hline
				\rowcolor[HTML]{32CB00} 
				Asset                                                                                                        & \begin{tabular}[c]{@{}l@{}}Threat/\\ Vulnerability\end{tabular}                          & \begin{tabular}[c]{@{}l@{}}Existing \\ Controls\end{tabular}                    & Likelihood                                                & Consequence & \begin{tabular}[c]{@{}l@{}}Level of \\ Risk\end{tabular} & \begin{tabular}[c]{@{}l@{}}Risk \\ Priority\end{tabular} \\ \hline
				\begin{tabular}[c]{@{}l@{}}Reliability and \\ integrity of the \\ SCADA nodes \\ and network\end{tabular}    & \begin{tabular}[c]{@{}l@{}}Unauthorized \\ modification\\ of control system\end{tabular} & \begin{tabular}[c]{@{}l@{}}Layered, \\ firewalls \\ and \\ servers\end{tabular} & Rare                                                      & Major       & High                                                     & 1                                                        \\ \hline
				\begin{tabular}[c]{@{}l@{}}Integrity of\\ stored file\\ and database\\ information\end{tabular}              & \begin{tabular}[c]{@{}l@{}}Corruption, theft,\\ loss of info\end{tabular}                & \begin{tabular}[c]{@{}l@{}}Firewall, \\ policies\end{tabular}                   & Possible                                                  & Major       & Extreme                                                  & 2                                                        \\ \hline
				\begin{tabular}[c]{@{}l@{}}Availability\\ and integrity\\ of financial\\ system\end{tabular}                 & \begin{tabular}[c]{@{}l@{}}Attacks/errors\\ affecting system\end{tabular}                & \begin{tabular}[c]{@{}l@{}}Firewall, \\ policies\end{tabular}                   & Possible                                                  & Moderate    & High                                                     & 3                                                        \\ \hline
				\begin{tabular}[c]{@{}l@{}}Availability\\ and integrity\\ of procurement\\ system\end{tabular}               & \begin{tabular}[c]{@{}l@{}}Attacks/errors \\ affecting system\end{tabular}               & \begin{tabular}[c]{@{}l@{}}Firewall, \\ policies\end{tabular}                   & Possible                                                  & Moderate    & High                                                     & 4                                                        \\ \hline
				\begin{tabular}[c]{@{}l@{}}Availability\\ and integrity\\ of maintenance/\\ production\\ system\end{tabular} & \begin{tabular}[c]{@{}l@{}}Attacks/errors \\ affecting system\end{tabular}               & \begin{tabular}[c]{@{}l@{}}Firewall, \\ policies\end{tabular}                   & Possible                                                  & Minor       & Medium                                                   & 5                                                        \\ \hline
				\begin{tabular}[c]{@{}l@{}}Availability,\\ integrity, and\\ confidentiality\\ of mail services\end{tabular}  & \begin{tabular}[c]{@{}l@{}}Attacks/errors \\ affecting system\end{tabular}               & \begin{tabular}[c]{@{}l@{}}Firewall, \\ ext mail \\ gateway\end{tabular}        & \begin{tabular}[c]{@{}l@{}}Almost \\ certain\end{tabular} & Minor       & High                                                     & 6                                                        \\ \hline
			\end{tabular}
		}
		\end{table}
	\end{frame}
	
	\begin{frame}
		\frametitle{Case Study: Conclusions}
		All of the resulting risk levels are above the acceptable minimum management specified as tolerable. \textbf{Hence treatment is required}:
		\begin{itemize}
			\item Management decided the first five risks should be treated
			by implementing suitable controls, which would reduce either the likelihood or the consequence should these risks occur. None of these risks could be accepted or avoided;
			\item Management decided that the risk to the SCADA network was
			unacceptable if there was any possibility of death, however remote;
			\item Responsability for the final risk	to the e-mail system was found to be primarily with the parent company’s IT group, which manages the external mail gateway. Hence the risk is shared with that group.
		\end{itemize}
	\end{frame}



	

	







	

	
\end{document}
